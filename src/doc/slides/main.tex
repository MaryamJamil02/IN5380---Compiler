% Created 2018-04-18 Wed 09:11
% Intended LaTeX compiler: pdflatex
\documentclass{beamer}
 		 

\input{switches}
\input{style/style-common}
\input{style/style-presentation}
%
%\newcommand{\deadlineone}   {Friday, 15. 03. 2019}
%\newcommand{\deadlineone}   {Friday, 06. 03. 2020}
\newcommand{\deadlineone}   {Friday, 12. 03. 2021}
%\newcommand{\deadlinetwo}   {--11. May  2018--}
%\newcommand{\deadlinetwo}   {--12. May  2019--}
%\newcommand{\deadlinetwo}    {12. May 2020}
\newcommand{\deadlinetwo}    {14. May 2021}



%%% Local Variables: 
%%% mode: latex
%%% TeX-master: "handout-oblig2.tex"
%%% End: 



\input{macros}
\input{macros-tmp}
\renewcommand{\maketitle}{}
\def\mytitle{Compiler Construction}    %% what for
\def\myauthor{Martin Steffen}   %% what for
\def\mydate{Spring 2018}        %% what for
\usetheme{default}
\usecolortheme{}
\usefonttheme{}
\useinnertheme{}
\useoutertheme{}
\date{Spring 2018}
\title{INF5110 -- Oblig 1}

\hypersetup{
 pdfauthor={},
 pdftitle={INF5110 -- Oblig 1},
 pdfkeywords={compiler, compiler construction},
 pdfsubject={compiler construction},
 pdfcreator={Emacs 24.3.1 (Org mode 9.1.6)}, 
 pdflang={English}}
\begin{document}

\maketitle
\chapter{Oblig 1}
\label{sec:org1292f40}



\section{Compila 18}
\label{sec:org50c56c4}

\begin{frame}[label={sec:orgbe9e676}]{Oblig 1}
\begin{itemize}
\item material (also for oblig 2) based on previous years, including contributions from Eyvind
W. Axelsen, Henning Berg, Fredrik Sørensen, and others
\end{itemize}


\begin{itemize}
\item see also the course web-page, containing links to ``resources''
\end{itemize}
\end{frame}


\begin{frame}[label={sec:org2e91fd9}]{Goal (of oblig 1)}
\begin{block}{Parsing}
Determine if programs written in \emph{Compila 18} are syntactically correct: 

\begin{itemize}
\item scanner
\item parser
\end{itemize}
\end{block}


\begin{itemize}
\item first part of a compiler, oblig 2 will add to it
\item language spec provided separatly
\end{itemize}
\end{frame}


\begin{frame}[label={sec:org6534cd4}]{Learning outcomes}
\begin{itemize}
\item using \alert{tools} for parser/scanner generation
\begin{itemize}
\item JFlex
\item CUP
\end{itemize}

\item variants of a grammar for the same languages

\begin{itemize}
\item \alert{transforming} one form (EBNF) to another (compatible with the used
tools)

\item controlling \alert{precedence} and \alert{associativity}
\end{itemize}

\item designing and implementing an \alert{AST}  data structure

\begin{itemize}
\item using the parsing tools to build such trees
\item pretty-printing such trees
\end{itemize}
\end{itemize}
\end{frame}


\begin{frame}[label={sec:org070139e}]{Compila language at a glance}

\lstinputlisting[basicstyle=\scriptsize]{code/myprogram.cmp}
\end{frame}



\begin{frame}[plain,label={sec:orga499f04}]{Another glance}

\lstinputlisting[basicstyle=\scriptsize]{code/swap.cmp}
\end{frame}


\begin{frame}[plain,label={sec:orgcd413aa}]{Grammar (1): declarations}

%~/cor/teaching/compila/src/doc/languagespec/
\lstinputlisting[basicstyle=\scriptsize,lastline=14]{../../doc/languagespec/grammar.txt}
\end{frame}






\begin{frame}[plain,label={sec:orge95e1ff}]{Grammar (2): declarations}

%~/cor/teaching/compila/src/doc/languagespec/
\lstinputlisting[basicstyle=\scriptsize,firstline=15,lastline=40]{../../doc/languagespec/grammar.txt}
\end{frame}




\begin{frame}[plain,label={sec:org3bdd3c7}]{Grammar (3): statements and types}

%~/cor/teaching/compila/src/doc/languagespec/
\lstinputlisting[basicstyle=\scriptsize,firstline=41]{../../doc/languagespec/grammar.txt}
\end{frame}




\section{Tools}
\label{sec:org8162c11}

\begin{frame}[fragile,label={sec:orga22a9ab}]{Tools: JFlex}
 \begin{itemize}
\item scanner generator (or lexer generator) tool

\begin{itemize}
\item \alert{input}: lexical specification
\item \alert{output}: scanner program in Java
\end{itemize}

\item lexical spec written as \texttt{.lex} file

\item consists of \alert{3 parts}

\begin{itemize}
\item user code
\item options and macros
\item lexical rules
\end{itemize}
\end{itemize}
\end{frame}


\begin{frame}[label={sec:orgd6bfb89}]{Sample lex code}
\includegraphics[width=\textwidth]{figures/snaps/lexcode}
\end{frame}



\begin{frame}[fragile,label={sec:org2b4f23d}]{CUP: Construction of useful parsers (for Java)}
 \begin{itemize}
\item a tool to easily (ymmv) generate \emph{parsers}

\item reads tokes from the scanner using \texttt{next\_token()}
\item the \texttt{\%cup} option (previous slide) makes that work
\end{itemize}



\begin{block}{Input}
grammar in BNF with \alert{action} code

\begin{verbatim}
var_decl ::= VAR ID:name COLON type:vtype
 {: RESULT =  new VarDecl(name, vtype); :};
\end{verbatim}
\end{block}


\begin{itemize}
\item \alert{output}: parser program (in Java)
\end{itemize}
\end{frame}


\begin{frame}[label={sec:orgbf01668}]{Sample CUP code}
\includegraphics[width=\textwidth]{figures/snaps/cupcode}
\end{frame}



\begin{frame}[fragile,label={sec:orga718865}]{Build tool: ant}

 \begin{center}
 \includegraphics[width=0.2\textwidth]{figures/snaps/antlogo}
\end{center}


\begin{itemize}
\item Java-based build tool (think ``make'')
\item config in \texttt{build.xml}
\item can contain different \alert{targets}
\end{itemize}

\begin{block}{typical general targets}
\begin{itemize}
\item test
\item clean
\item build
\item run
\end{itemize}
\end{block}

\begin{itemize}
\item supplied configuration should take care of calling \texttt{jflex}, \texttt{cup}, and
\texttt{javadoc} for you
\end{itemize}
\end{frame}

\begin{frame}[label={sec:orgbec441f}]{AST data structure}

\begin{center}
 \includegraphics[width=0.8\textwidth]{figures/snaps/astclasses}
\end{center}
\end{frame}




\begin{frame}[label={sec:org454aac9}]{Provides source code}

\begin{center}
 \includegraphics[width=0.8\textwidth]{figures/snaps/structureoverview}
\end{center}
\end{frame}


\begin{frame}[label={sec:org802efc9}]{Building: putting it together}

\begin{center}
 \includegraphics[width=0.8\textwidth]{figures/snaps/buildprocess}
\end{center}
\end{frame}

\section{Official}
\label{sec:org71545b7}

\begin{frame}[fragile,label={sec:org4f0d505}]{Deadline}
 \begin{alertblock}{Deadline}
\alert{Friday 23. 03. 2018, 23:59}
\end{alertblock}


\begin{itemize}
\item don't miss the deadline
\item for extensions, administration needs to agree (\texttt{studadm}), contact them
if sick etc
\item even if not 100\% finished
\begin{itemize}
\item deliver what you have
\item contact early when problems arise
\end{itemize}
\end{itemize}
\end{frame}


\begin{frame}[fragile,label={sec:org014657d}]{Deliverables}
 \begin{itemize}
\item see also the ``handout''
\end{itemize}

\begin{block}{Deliverables (1)}
\begin{itemize}
\item working \alert{parser}
\begin{itemize}
\item parse the supplied sample programs
\item printout the resulting AST
\end{itemize}

\item \alert{two} grammars (two \texttt{.cup}-files)

\begin{itemize}
\item one unambiguious
\item one ambiguous, where ambibuities resolved through precedence
declations in \emph{CUP}, e.g.
\end{itemize}
\end{itemize}

\begin{verbatim}
precendence left AND;
\end{verbatim}
\end{block}
\end{frame}


\begin{frame}[label={sec:org869ed22}]{Deliverables}
\begin{block}{Deliverables (2)}
\begin{itemize}
\item report (with name(s) and UiO user name(s)
\item discussion of the solution (see handout for questions)
\item in particular: comparison of the two grammars
\item ``Readme''
\end{itemize}
\end{block}

\begin{itemize}
\item the code must \emph{build} (with ant) and run
\item test it on the  UiO RHEL platform
\end{itemize}

\begin{block}{Ask}
If problems, \alert{ask in time}  (\alert{NOT} Friday at the deadline)
\end{block}
\end{frame}



\begin{frame}[fragile,label={sec:org72b5c30}]{Hand-in procedure}
 \begin{itemize}
\item this year we try \emph{git}

\item \url{https://github.uio.no} resp. \url{https://github.uio.no/msteffen/compila}

\item you need

\begin{itemize}
\item a login
\item send me emails that you want to do oblig (+ potential partner)
\(\Rightarrow\) I tell you group number
\item create a project \texttt{compila<n>} (\(n\) = group number)
\item add collaborator + (at some point me)
\end{itemize}
\end{itemize}


\begin{itemize}
\item see also the handout
\item code ready \emph{tomorrow}
\end{itemize}
\end{frame}





\chapter{Oblig2}
\label{sec:org9eb6ae6}


\begin{frame}[fragile,label={sec:org01d2262}]{Goal}
 \begin{enumerate}
\item \alert{semantic  analysis}, as far as 
\begin{itemize}
\item \alert{typing} is concerned (``static semantics'')
\item other coditions (no duplicate declaration etc)
\end{itemize}

\item \alert{code generation} for \texttt{compila18} (ish) programs
\end{enumerate}
\end{frame}



\section{References}
\label{sec:org884b755}

\begin{frame}[allowframebreaks]
  \frametitle{References}
  {\tiny
    \bibliographystyle{apalike}
%    \bibliography{extracted,local} 
    \bibliography{string,semantics,etc,mobile,crossref}  %general PMA
  }
\end{frame}




%%%%%%%%%%%%%%%%%%%%%%%%%%%%%%%%%%%%%%%%%%%%%%%%%%%%%%%%%%%%%%%%%%%%%%%%%%%%%%%%%
%%% Local Variables: 
%%% mode: latex
%%% TeX-master: t
%%% End: 
\end{document}