%%%%%%%%%%%%%%%%%%%%%%%%%%%%%%% macros %%%%%%%%%%%%%%%%%%%%5


\newcommand{\Cplusplus}                    {C$^{++}$}
\newcommand{\Csharp}                       {C$^\sharp$}
\newcommand{\Tinylang}                     {TINY}

\ifhandout
%\renewcommand{\bcell}{}  %% something's wrong here
\fi


\newcommand\doubleplus{+\kern-1.3ex+\kern0.8ex}
\newcommand\mdoubleplus{\ensuremath{\mathbin{+\mkern-10mu+}}}

\ifhandout
\else
\renewcommand<>{\emph}[1]{{\only#2{\itshape}\blue{#1}}}
\fi



\newcommand{\Aut}                          {\mathcal{A}}
\newcommand{\Lang}                         {\mathcal{L}}
\newcommand{\midtilde}{\raisebox{0.5ex}{\texttildelow}}

\newcommand{\many}                         {*}
\newcommand{\p}                            {\vdash}
\newcommand{\T}                            {T}
\newcommand{\Gp}                           {\Gamma\p}
\newcommand{\OFF}                          {::}

%%%%%%%%%%%%%%%%%%% tac %%%%%%%%%%%%%%%%%%%%%%%%%%%%%%%%%%%%
\newcommand{\tatf}[1]                      {\mathop{\mathbf{#1}}}
\newcommand{\taf}[1]                       {\mathtt{#1}}
\newcommand{\taread}                       {\mathop{\taf{read}}}
\newcommand{\tawrite}                      {\mathop{\taf{write}}}
\newcommand{\taiffalse}                    {\mathop{\taf{if\_false}}}
\newcommand{\tagoto}                       {\mathop{\taf{goto}}}
\newcommand{\talabel}                      {\mathop{\taf{label}}}
\newcommand{\taand}                        {\mathop{\taf{and}}}
\newcommand{\taor}                         {\mathop{\taf{or}}}
\newcommand{\taop}                         {\mathop{\tatf{op}}}

%%%%%%%%%%%%%%%%%%% pcode %%%%%%%%%%%%%%%%%%%%%%%%%%%%%%%%%%%%
\newcommand{\ptf}[1]                      {\mathop{\mathbf{#1}}}
\newcommand{\pf}[1]                       {\mathtt{\mathbf{#1}}}

\newcommand{\agpf}{\gatt{pf}}
\newcommand{\agstrval}{\gatt{strval}}
\newcommand{\gapcode}{\gatt{pcode}}
\newcommand{\gastrval}{\gatt{strval}}

\newcommand{\pcodelit}[1]{\red{"#1"}}


\newcommand{\plda}                        {\pf{lda}} 
\newcommand{\pstn}                        {\pf{stn}}
\newcommand{\psto}                        {\pf{sto}}
\newcommand{\padi}                        {\pf{adi}} 
\newcommand{\pldc}                        {\pf{ldc}} 
\newcommand{\plod}                        {\pf{lod}} 
\newcommand{\pujp}                        {\pf{ujp}}
\newcommand{\pfjp}                        {\pf{fjp}}
\newcommand{\plab}                        {\pf{lab}}
\newcommand{\pneq}                        {\pf{neq}}
\newcommand{\pequ}                        {\pf{equ}}
\newcommand{\pcconc}                      {\doubleplus}
\newcommand{\pcconcone}                   {\textasciicircum}
\newcommand{\ganame}                      {\gatt{name}}



%%%%%%%%%%%%%%%%% three-address code

\newcommand{\ptactf}[1]                     {\mathop{\mathbf{#1}}}
\newcommand{\tacf}[1]                       {\mathtt{\mathbf{#1}}}
\newcommand{\taclit}[1]                     {\red{"#1"}}
\newcommand{\tacassign}                     {\tacf{=}}
%\renewcommand{\tacassign}                    {\leftarrow}
\newcommand{\tacplus}                       {\tacf{+}}
\newcommand{\tacempty}                      {\tacf{}}
\newcommand{\gatacode}                    {\gatt{tacode}}
\newcommand{\newtemp}                     {\mathit{newtemp}} 


\newcommand{\rmgoto}                      {\mathop{\mathbf{GOTO}}}

%%%%%%%%%%%%%%%%%%% regular expressions 
\newcommand{\renonterm}                    {r}
\newcommand{\renontermr}                   {r}
\newcommand{\renonterms}                   {s}
\newcommand{\re}[1]                        {\boldsymbol{#1}}
\newcommand{\rechar}[1]                    {\mathbf{#1}}
\newcommand{\renot}                        {\re{\midtilde}}
\newcommand{\reemptyword}                  {\re{\emptyword}}
\newcommand{\reemptyset}                   {\re{\emptyset}}
\newcommand{\remany}                       {{\re{\many}}}
\newcommand{\reoneormore}                  {{\re{+}}}
\newcommand{\reor}                         {\re{\mid}}
\newcommand{\reopt}                        {\re{?}}
\newcommand{\reempty}                      {\re{\emptyset}}
\newcommand{\reall}                        {\mathbf{.}}

%%%%%%%%%%%%%%%%%%%%%%%%%%%%%%%%%%%%%%%%%%%%%%%%%%%%%%%%%%%%%%%%%%%%%
\newcommand{\Asucc}[2]                     {#1_{#2}}
\newcommand{\Adet}[1]                      {\overline{#1}}
\newcommand{\Aeclosure}                    {\mathit{close}_{\epsilon}}    %% epsilon-closure
\newcommand{\Aeclosureof}[1]               {\Aeclosure(#1)}



\newcommand{\gasetto}                      {=}
%\renewcommand{\gasetto}                    {\leftarrow}
\newcommand{\gts}[1]                        {\mathop{\boldsymbol{#1}}}  %% not nice
\newcommand{\gt}[1]                        {\mathit{\boldsymbol{\mathrm{#1}}}}  %% not nice
%\newcommand{\gt}[1]                        {\mathbf{#1}}  %% not nice
\newcommand{\gatt}[1]                      {\mathtt{#1}}
\newcommand{\gnt}[1]                       {\mathop{\mathit{#1}}}   %% mathop: subscripts not good
\newcommand{\bnfprod}                      {\rightarrow}
\newcommand{\bnfmany}[1]                   {\{\; #1  \;\}}
\newcommand{\Gdstep}                       {\Rightarrow}
\newcommand{\Gdstepwith}[1]                {\Rightarrow_{#1}}
\newcommand{\Gdstepreverse}                {\Leftarrow}
\newcommand{\Gdstepl}                      {\Gdstep_l}        %% left derivation
\newcommand{\Gdsteplmany}                  {\Gdstepmany_l}        %% left derivation
\newcommand{\Gdstepr}                      {\Gdstep_r}        %% right derivation
\newcommand{\Gdsteprreverse}               {\Gdstepreverse_r}        %% right derivation
\newcommand{\Gdsteprmany}                  {\Gdstepmany_r}        %% right derivation
\newcommand{\Gdstepmany}                   {\Rightarrow^{\many}}
\newcommand{\Gdstepmanyusing}[1]           {\Rightarrow_{#1}^{\many}}
\newcommand{\Gdstepusing}[1]               {\Gdstep_{#1}}

\newcommand{\Gred}{\hookrightarrow}       

%%% grammars
\newcommand{\gtoken}[2] {\angles{{\mathtt{#1}, #2}}}
\newcommand{\gtokenname}[1] {\mathtt{#1}}
\newcommand{\gtokenatt}[1]  {#1}


\newcommand{\Gfirst}                       {\mathit{First}}
\newcommand{\Gfollow}                      {\mathit{Follow}}

\newcommand{\Galphabet}                    {\Alphabet}
\newcommand{\Gterms}                       {\Alphabet_T}
\newcommand{\Gnonterms}                    {\Alphabet_N}
\newcommand{\Gprods}                       {P}
\newcommand{\Gstart}                       {S}
\newcommand{\ghandle}[2]                   {\angles{#1,#2}}

\newcommand{\gemptyword}                   {\boldsymbol{\emptyword}}
\newcommand{\gtword}                       {w}
\newcommand{\gtwordw}                      {w}
\newcommand{\gtwordv}                      {v}
\newcommand{\gword}                        {\alpha}     % terminals & non-terminals, also called string
\newcommand{\gworda}                       {\alpha}     % terminals & non-terminals, also called string
\newcommand{\gwordb}                       {\beta}      % terminals & non-terminals, also called string
\newcommand{\gwordc}                       {\gamma}     % terminals & non-terminals, also called string
\newcommand{\gworde}                       {\eta}       % terminals & non-terminals, also called string
\newcommand{\gsymbol}                      {X}
\newcommand{\gsymboly}                     {Y}
\newcommand{\gnonterm}                     {A}
\newcommand{\gnonterma}                    {A}
\newcommand{\gnontermb}                    {B}
\newcommand{\gnontermc}                    {C}
\newcommand{\gendmark}                     {\gts{\$}}
\newcommand{\parsepos}                     {\gt{.}}



\newcommand{\first}                        {\mathit{first}}

\newcommand{\sclos}[2]                     {\angles{#1,#2}}

\newtheorem{proposition}{Proposition}


\ifhandout
\newcommand{\hofootnote}[1]{\footnote{#1}}

\else

\newcommand{\hofootnote}[1]{}
\fi


\newcommand{\Sig}                            {\Sigma}
\newcommand{\Alphabet}                       {\Sigma}
\newcommand{\Astates}                        {Q}
\newcommand{\Aistates}                       {I}
\newcommand{\Afstates}                       {F}
\newcommand{\Atransrel}                      {\delta}
\newcommand{\Atransfun}                      {\delta}
\newcommand{\Atrans}[1]                      {\trans{#1}}

\newcommand{\emptyword}                      {\epsilon}




%%%% Some general math 
\newcommand{\eqdef}                          {\triangleq}
\newcommand{\eqisit}                         {=^?}
\newcommand{\redex}[1]                       {\underline{#1}}
\newcommand{\eat}[1]                         {\text{\sout{\ensuremath{#1}}}}  %%% have to be careful with math
\newcommand{\substfor}[2]                    {{}_{#2 \leftarrow #1}}
\renewcommand{\substfor}[2]                  {[#1/#2]}
\newcommand{\OF}                             {\mathrel{:}}
\newcommand{\of}                             {:}
\ifhandout\else
\renewcommand{\complement}[1]                {\overline{#1}}
\fi
\newcommand{\intersect}                      {\cap}
\newcommand{\union}                          {\cup}
\newcommand{\dunion}                         {+}
\newcommand{\conc}                           {\circ}
\newcommand{\sconc}{\parallel}
\newcommand{\without}                        {\backslash}


\newcommand{\bnfdef}               {::=}
\newcommand{\bnfbar}               {\ \mathrel{|}\ }
%%% % logic, start with l, such as \land and \lor
%%
\newcommand{\ltop}                         {\top}
\newcommand{\lbot}                         {\bot}
\newcommand{\ltrue}                        {\mathit{true}}
\newcommand{\lfalse}                       {\mathit{false}}
\newcommand{\limplies}                     {\rightarrow}
\newcommand{\limpliesback}                 {\leftarrow}
\newcommand{\lequivalent}                  {\lequiv}  %%
\newcommand{\lequiv}                       {\leftrightarrow}
\newcommand{\ltest}[1]                     {#1?}

%% LTL
\newcommand{\luntil}                       {\mathit{U}}
\newcommand{\lrelease}                     {\mathit{R}}
\newcommand{\lwaitingfor}                  {\mathit{W}}
\newcommand{\lwaitingforp}                 {\lwaitingfor^{-1}}
\newcommand{\lnext}                        {\bigcirc}

\newcommand{\until}[2]                     {#1 \luntil #2}
\newcommand{\release}[2]                   {#1 \lrelease #2}


\newcommand{\lalways}                      {\Box}%% {\oblong}
\newcommand{\leventually}                  {\Diamond}
\newcommand{\lalwaysp}                     {\lalways^{-1}}
\newcommand{\leventuallyp}                 {\leventually^{-1}}
%\DeclareMathOperator{\hf}{\oblong}  %% henceforth

%%%  modal logics
\newcommand{\lbox}                         {\Box}

\newcommand{\ldiamond}                     {\Diamond}
%%%
\newcommand{\sem}[1]                       {{#1}^I}
\newcommand{\sat}                          {\models}

%%%%  Hoare logic
\newcommand{\ssat}                         {\mid\!\models}   %% do we need that?

%%%% 
\newcommand{\rmcontents}                   {\mathit{contents}}

\newcommand{\dlbox}[1]                     {[#1]}            %% dynamic logics}
\newcommand{\dldiamond}[1]                 {\langle#1\rangle}
%%% t for temporal
\newcommand{\tpath}                        {\sigma}          %%  path conflicts with tikz
\newcommand{\tsat}                         {\models}
\newcommand{\lrstate}[2]                   {{#1}_{#2}}
\newcommand{\lroneitem}[2]                 {[#1,#2]}

%%%%%%%%%%%%%%%%%  %%%%%%%%%%%%%%%%%%%%%%%%%%%%%%%%%%%%
\newcommand{\gnode}[1]{$\begin{aligned}\textstyle#1\end{aligned}$}
%%%%%%%%%%%%%%%%%  %%%%%%%%%%%%%%%%%%%%%%%%%%%%%%%%%%%%
%%%%%%%%%%%%%%%%%%%%%%%%%% inf5140.sty + some others
\newcommand{\emptyseq}                      {\epsilon}
\newcommand{\terminated}                    {\mathit{terminated}}
\newcommand{\lang}[1][]{\mathcal{L}_{#1}}
\newcommand{\langfol}                       {\lang^{1}}
\newcommand{\langtl}                        {\lang[T]}

\newcommand{\atstate}[1]                    {\mathit{at}(#1)}


\newcommand{\bindsto}[2]                   {#1 \mapsto #2}
\newcommand{\stenv}                        {\sigma}
\newcommand{\stlookup}                     {\mathit{lookup}}
\newcommand{\stdelete}                     {\mathit{delete}}
\newcommand{\stinsert}                     {\mathit{insert}}
\newcommand{\stdecl}                       {\mathit{decl}}
\newcommand{\stname}                       {\mathit{name}}

% %\newcommand{\xyseq}[1]{%
%   \begin{figure}[H]
%     \xymatrixcolsep{5ex}
%     \xymatrixrowsep{-1ex}
%     \fbox{\xymatrix{#1}}
% \end{figure}}
\newcommand{\tlnode}[2] {\bullet^{#1}_{#2}}



%%%% PGF 3.0
\tikzset{mycomplexfigure/.pic={
        \draw[rounded corners] (0,0) rectangle (5,2);
        \draw (1,1) circle (0.3);
        \draw (4,1) circle (0.3);
        \node at (2.5,1.0)  {$\ldots$};
        \node at (2.5,1.2)  {$r$};
    }
}


\tikzset{
    pics/thompson/.style args={#1,#2,#3,#4}{
        code={
        \draw[rounded corners, very thick] (0,0) rectangle (5,2);
        \node (#1) [state] at (1,1) {};
        \node (#3) [state,#4] at (4,1) {};
%        \draw (1,1) circle (0.3);
%        \draw (4,1) circle (0.3);
        \node at (2.5,1.0)  {$\ldots$};
        \node at (2.5,1.2)  {#2};
%        \node (#2) at (1,1) {X};
    }}}

\tikzset{
    pics/mysymbol/.style args={#1 scale #2 with #3}{
        code={\node at (1,1) {};}}}




\newcommand{\tlgraph}{
  \begin{tikzpicture}[descr/.style={fill=white,inner sep=2.5pt}]
    \matrix (m) [matrix of math nodes,row sep=3em,column sep=2.5em,text height=1.5ex,text depth=0.25ex]
    { 
        \tlnode{1}{}
        & 
        \tlnode{2}{}
        & 
        \tlnode{3}{}
        & 
        \tlnode{4}{}
        & 
        \tlnode{5}{}
        &
        \ldots
        \\   %% new line is necessary
      }
      ; 
      \path[->,font=\scriptsize]
      (m-1-1) edge (m-1-2) 
      (m-1-2) edge (m-1-3)
      (m-1-3) edge (m-1-4)
      (m-1-4) edge (m-1-5)
      (m-1-5) edge (m-1-6)
      ;
    \end{tikzpicture}
}



%%%%%%%%%%%%%%%%%%%%%% for CFG


\tikzset{
    cfgnode/.style={
           rectangle,
           fill=black!04,
           rounded corners,
           draw=black, very thick,
           minimum height=2em,
           inner sep=2pt,
           text centered,
           },
}


\tikzstyle{cfgedge} = [->,very thick,black]

\newcommand{\codenode}[3]{
  \begin{tabular}[t]{l}
    #2
  \end{tabular}
}


%%%%%%%%%%%%%%%%%%%%%%%% for dags 


\tikzstyle{dagedge} = [-,very thick,black]
\tikzset{
    dagopnode/.style={
           fill=black!10,
           circle,
           rounded corners,
           draw=black, very thick,
           minimum height=2em,
           inner sep=2pt,
           text centered,
           },
}


\tikzset{
    dagdeadopnode/.style={
           fill=red!25,
           circle,
           rounded corners,
           draw=black, very thick,
           minimum height=2em,
           inner sep=2pt,
           text centered,
           },
}

\tikzset{
    dagliveopnode/.style={
           fill=green!25,
           circle,
           rounded corners,
           draw=black, very thick,
           minimum height=2em,
           inner sep=2pt,
           text centered,
           },
}


\tikzset{
    dagleafnode/.style={
           circle,
           fill=black!3,
           rounded corners,
           draw=black,%dotted,
           minimum height=2em,
           inner sep=2pt,
           text centered,
           },
}


%\renewcommand{\xyseq}[1]{---}
%%%%%%%%%%%%%%%%%%%%%%%%%%%%% not done yet





%\newcommand{\leadsfromto}                         {???}  %%% \triple
%\newcommand{\corrp}                         {?}
%\newcommand{\corrt}                         {?}
\newcommand{\Nat}                           {-}
\newcommand{\entails}{???\Rightarrow??}
\newcommand{\stl}  {~s.th.~}
\renewcommand{\powerset}[1]                   {2^{#1}}

\newcommand{\scoord}[2]                    {\angles{#1,#2}}   %% static coordinate


\newcommand{\kw}[1]                           {\mathtt{#1}}

\newcommand{\Tok}                             {\kw{ok}}
\newcommand{\Treal}                           {\kw{real}}
\newcommand{\Tbool}                           {\kw{bool}}
\newcommand{\Tvoid}                           {\kw{void}}
\newcommand{\Tint}                            {\mathrel{\kw{int}}}
\newcommand{\Tarray}[2]                       {\kw{array} #1 \kw{of} #2}
\newcommand{\Terror}                          {\mathrel{\kw{error}}}
\newcommand{\Tunion}                          {\mathrel{+}}

\newcommand{\ifs}                             {\mathrel{\kw{if}}}
\newcommand{\ins}                             {\mathrel{\kw{in}}}
\newcommand{\nis}                             {\mathrel{\kw{ni}}}
\newcommand{\thens}                           {\mathrel{\kw{then}}}
\newcommand{\elses}                           {\mathrel{\kw{else}}}
\newcommand{\fis}                             {\mathrel{\kw{fi}}}
\newcommand{\whiles}                          {\mathrel{\kw{while}}}
\newcommand{\dos}                             {\mathrel{\kw{do}}}
\newcommand{\fors}                            {\mathrel{\kw{for}}}
\newcommand{\tos}                             {\mathrel{\kw{to}}}
\newcommand{\trues}                           {\kw{true}}
\newcommand{\falses}                          {\kw{false}}
\newcommand{\conds}                           {\kw{cond}}
\newcommand{\calls}                           {\kw{call}}
\newcommand{\signals}                         {\kw{signal}}
\newcommand{\signalalls}                      {\kw{signal\_all}}
\newcommand{\waits}                           {\kw{wait}}
\newcommand{\emptys}                          {\kw{empty}}
\newcommand{\choices}                         {[\;]}
\newcommand{\sendatos}[2]                     {\kw{send}\ #1:#2}

\newcommand{\precond}[1]                      {\{\;#1\;\}}
\newcommand{\postcond}[1]                     {\{\;#1\;\}}
\newcommand{\cond}[1]                         {\{\;#1\;\}}
\newcommand{\triple}[3]                       {\cond{#1}\; #2\; \cond{#3}}
\newcommand{\ctriple}[3]                      {\triple{#1}{\mathtt{#2}}{#3}}  %%% concrete triple: code in tt

\newcommand{\awaits}                          {\kw{await}}
\newcommand{\skips}                           {\kw{skip}}
\newcommand{\await}[2]                        {\atomic{\awaits (#1)\ #2}}
\newcommand{\sends}                           {\kw{send}}





\ifslides
\newcommand{\important}[1]{{\blue{#1}}}
\else
\providecommand{\important}[1]{\colorbox{yellow}{#1}}
\renewcommand{\important}[1]{{\blue{#1}}}
\fi
\newcommand{\importantx}[1]{\red{#1}}


\newcommand{\numberofprocs}[1]   {\# #1}


%%%%% Liveness (intra-block & also global)

\newcommand{\inlive}                    {\mathit{inLive}}
\newcommand{\outlive}                   {\mathit{outLive}}
\newcommand{\samedesc}                  {$\cdot$}
\newcommand{\canceldesc}[1]             {\cancel{#1}}
\newcommand{\tempdesc}[1]               {[#1]}                  %% ? 
\newcommand{\getreg}                    {\mathit{getreg}}
\newcommand{\codegen}                   {\mathit{codegen}}
\newcommand{\livenextlocal}[1]          {L(#1)}  
%\newcommand{\livenextnonlocal}         {L(\downVdash)}
\newcommand{\livenextnonlocal}          {L(\bot)}
%\newcommand{\livenextdeadlocal}        {\bot}
%\newcommand{\livenextdeadnonlocal}     {\upVdash}
\newcommand{\livenextdeadlocal}         {D}
\newcommand{\livenextdeadnonlocal}      {D}


%%%%% code generation
\newcommand{\locsof}[2]        {#2(#1)}
\newcommand{\tablerd}          {T_{\mathit{r}}}
\newcommand{\tablelive}        {T_{\mathit{live}}}
\newcommand{\tableliveat}[2]   {\tablelive[#1,#2]}
\newcommand{\tablead}          {T_{\mathit{a}}}
\newcommand{\setto}[2]         {[#1 \mapsto #2]}
\newcommand{\setaddto}[2]      {[#1 \mapsto_{{\union}}  #2]}
\newcommand{\setwithoutto}[2]  {\without (#1 \mapsto  #2)}
\newcommand{\Reg}              {\mathit{Reg}}
\newcommand{\Id}               {\mathit{Id}}
\newcommand{\Loc}              {\mathit{Loc}}
\newcommand{\Toption}[1]       {#1_{\undef}}
%%%%%%%%%%%%%%%%%%%%%%%%5%%%%%%


\newenvironment{ruleset}{
  \noindent\mbox{}\noindent\hrulefill
  \begin{displaymath}
    \renewcommand{\arraystretch}{1.3}\begin{array}[b]{l}}{
  \end{array}\end{displaymath}\hrulefill\mbox{}}

\newenvironment{rulesetnolines}{
  \mbox{}\noindent\begin{displaymath}
    \renewcommand{\arraystretch}{1.3}\begin{array}[b]{l}}{
  \end{array}\end{displaymath}\mbox{}}

\newenvironment{ruleenv}{\begin{displaymath}}{\end{displaymath}}

\newcommand{\ruleskip}   {\\ \\[-1.1em]}

\newcommand{\axskip}   {\\[-0.3em]}

\newcommand{\treefont}{\small}
\newcommand{\leaf}[1]{{\begin{array}{c} #1 \end{array}}}
\newenvironment{proofleaf}{\begin{array}[t]{c}}{\end{array}}

\newenvironment{treeenv}{\begin{displaymath}}{\end{displaymath}}


\newcommand{\namedruletree}[3]{{
     \treefont\textstyle
     \prooftree 
       {\leaf{#2}}
     \justifies
     \textstyle #3
     \using \rn{#1}
     \endprooftree
     }}                             

\newcommand{\namedaxiomtree}[2]{{
     \treefont
     \prooftree 
       #2
     \using \rn{#1}
     \endprooftree
     }}

\newcommand{\ruletree}[2]{\namedruletree{}{#1}{#2}}




\newcommand{\qqquad}  {\quad\quad\quad}
\newcommand{\andalso}{\quad\quad}
\newcommand{\infaxiom}[2]{\namedaxiomtree{#1}{#2}}
\newcommand{\infrule}[3]{\namedruletree{#1}{#2}{#3}\andalso}
\newcommand{\scinfrule}[4]{\namedruletree{\mbox{$#4$}\quad\quad #1}{#2}{#3}\andalso}
\newcommand{\infax}[2]{ %%% old
  \treefont
  #2\quad\quad \rn{#1}\andalso
  \andalso
  }


\renewcommand{\infax}[2]{
  {\mbox{\treefont \ensuremath{#2}}}\quad\treefont\rn{#1}\andalso\andalso}

\newcommand{\scinfax}[3]{
  \treefont
  #2\quad \mbox{$#3$\quad#1}
  \andalso
}

\newcommand{\rn}[1]{\mbox{\textsc{#1}}}        %% rule name

\newcommand{\pre}{\mathit{pre}}



\newcommand{\transpose}[1]                  {#1^t}

\newcommand{\suchthat}                      {\ | \ }
\newcommand{\unimportant}[1]                {\textcolor{gray}{#1}}
%\newcommand{\falses}                        {F}
\newcommand{\undefs}                         {\bot}   %% // undef causes errors, renew as well
\newcommand{\sizeof}[1]                     {|#1|}
\newcommand{\lengthof}[1]                   {|#1|}
\newcommand{\R}                             {\mathbb{R}}
\newcommand{\set}[1]                        {\{#1\}}
\newcommand{\angles}[1]                     {\langle #1 \rangle}

\newcommand{\bigO}                          {\mathcal{O}}

\newcommand{\dist}                          {d}
\newcommand{\weight}                        {w}
\newcommand{\transmany}[1]                  {\longrightarrow^{\ast}}
\newcommand{\trans}[1]                      {\xrightarrow{#1}}
\newcommand{\Asem}[1]                       {\mathcal{L}(#1)}
%\newcommand{\to}                            {\rightarrow}




\definecolor{beamerblue}{rgb}{0.2,0.2,0.7}  %% found in beamer.cls
\definecolor{beamerred}{rgb}{0.7,0.2,0.2}   
\definecolor{beamergreen}{rgb}{0.0,0.40,0}    % dark green

%red, green, blue, yellow, cyan, magenta, black, white
\newcommand{\blue}[1]{\textcolor{beamerblue}{#1}}
\newcommand{\red}[1]{\textcolor{beamerred}{#1}}
\newcommand{\green}[1]{\textcolor{beamergreen}{#1}}%forrestgreen
\definecolor{rawsienna}{cmyk}{0,0.72,1,0.45}
\newcommand{\rawsienna}[1]{\textcolor{rawsienna}{#1}}


%%%%%%%%%%%%%%%%%%%%%%%%%%%OLD



\newcommand{\pstate}[1] {\angles{#1}}


%\let\myem=\em
%\renewcommand{\em}{\myem\color{red}}

\newcommand{\header}[1]{\Blue{\Large{\bf #1}}}


\newcommand{\ra}{\ensuremath{\rightarrow}}

%% NOOOOOO \newcommand{\mybox}[1]{\epsfig{file=figures/boks.eps,scale=.6}\hspace{-15pt}\raisebox{2mm}{\ensuremath{#1}}\hspace{1mm}}




\ifslides\else
%\newtheorem{example}             {Example}
\fi
\newenvironment{answer}{%
  \trivlist 
\item[\hskip \labelsep{\bf Answer:
    }]%
  }{\endtrivlist}
\newtheorem{question}           {Question} [section] 
\ifslides\else
%\newtheorem{example}            {Example} [question] 
\fi


\makeatletter
\newif\if@qeded\global\@qededfalse
\def\proof{\global\@qededfalse\@ifnextchar[{\@xproof}{\@proof}}
\def\endproof{\if@qeded\else\qed\fi\endtrivlist}
\def\qed{\unskip \hfill{\unitlength1pt\linethickness{.4pt}\framebox(6,6){}}
\global\@qededtrue}
\def\@proof{\noindent\trivlist \item[\hskip %10pt\hskip 
 \labelsep{\bf Proof:}]\ignorespaces}
\def\@xproof[#1]{\trivlist \item[\hskip %10pt\hskip 
 \labelsep{\bf Proof #1:}]\ignorespaces}
\makeatother

%%%%%%%%%%%%%%%%% draw stack hacks

%% the following is a change of \cellprt. The pointer now points not in the
%% middle of the cell, but to the ``start''. That coincides to the
%% representation in the book. It's achieved by inserting a ``y-offset'' of
%% "0.5+..." to the pointer and the annotation. I also made a
%% cell-end-pointer, for convenience (the pointers only work on
%% standard-sized cells).

%% In the picts, the real pointer is supposed the end pointer?

\newcommand{\cellstartptr}[1]{
  \draw[<-,line width=0.7pt] (0,\value{cellnb}) +(2,0.5+\value{ptrnb}*0.1) -- +(2.5,0.5+\value{ptrnb}*0.45);
  \draw (2.5,0.5+\value{ptrnb}*0.5+\value{cellnb}) node[anchor=west] {#1};
  \addtocounter{ptrnb}{1}
}

\newcommand{\cellendptr}[1]{
  \draw[<-,line width=0.7pt] (0,\value{cellnb}) +(2,-0.5+\value{ptrnb}*0.1) -- +(2.5,-0.5+\value{ptrnb}*0.45);
  \draw (2.5,-0.5+\value{ptrnb}*0.5+\value{cellnb}) node[anchor=west] {#1};
  \addtocounter{ptrnb}{1}
}


\newcommand{\cellrealptr}[1]{\cellendptr{#1}}

%%% Local Variables: 
%%% mode: latex
%%% TeX-master: "main"
%%% End: 
