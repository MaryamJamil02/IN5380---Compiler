\documentclass[11pt,freeform]{handout}[2014/08/13]

%\dozent{\mbox{}}
\def\myauthor{\href{https://martinsteffen.github.io}{Martin Steffen}}
\def\mytitle{INF 5110: Compiler construction}

\coursename{INF 5110: Compiler construction}
\semester{Spring 2021}
\dozent{Martin Steffen}
\ifigroup{Reliable systems}


%%% Local Variables: 
%%% mode: latex
%%% TeX-master: t
%%% End: 

%
%\newcommand{\deadlineone}   {Friday, 15. 03. 2019}
\newcommand{\deadlineone}   {Friday, 06. 03. 2020}
%\newcommand{\deadlinetwo}   {--11. May  2018--}
%\newcommand{\deadlinetwo}   {--12. May  2019--}
\newcommand{\deadlinetwo}    {12. May 2020}



%%% Local Variables: 
%%% mode: latex
%%% TeX-master: "handout-oblig1.tex"
%%% End: 



\pgfdeclareimage[height=2.4cm,interpolate=true]{uio}{logos/uiologo}%% relative
%\usepackage[german]{babel} 
%\usepackage{german}

\handouttitle{Oblig 2}
\handoutnumber{2} 
%\issuedate{18.\ 04.\ 2018}
\issuedate{10.\ 04.\ 2019}
\issuedate{2.\ 04.\ 2020}
\issuedate{7.\ 04.\ 2021}
\topic{Oblig 2}




\begin{document}
\thispagestyle{empty}



\section{Official }
\label{sec:official-info}




\hrulefill{}

The \textbf{deadline/frist}  for the second oblig is

\begin{quote}
  \textbf{\deadlinetwo}
\end{quote}





\section{What and how to hand in}
\label{sec:what-how}


\subsection{Git}
\label{sec:git}

You will continue with your group's git-repos you used in the first oblig
(unless really convincing reasons speak against it).  Basically, you
continue with your previous code, add the new functionality, push a
solution before the deadline, and inform me when it's done so that I can
update. It's important to tell me, as I don't want to repeatedly update in
the hope that it's done.


If a change in arrangement is needed (merge of groups, or a split of
groups), you need to ask for that re-arrangement (not just that on the day
of the deadline it's ``announced'' that there is now a new group \ldots).



See also the \emph{Readme} of the \emph{``patch''} under

\begin{quote}
  \url{https://github.uio.no/msteffen/compila/tree/master/oblig2patch}
\end{quote}


\subsection{What to include into a solution}
\label{sec:what-include-into}

As before, it should be an appropriately commented repos, solving the tasks
of oblig 2. In particular needed is (basically as before)


\begin{itemize}
\item A top-level \emph{Readme-file}\footnote{Many did a \texttt{Readme.md}
    which is a good format.} containing
  \begin{itemize}
  \item  containing names and emails of the authors
  \item instructions how to build the compiler and how to run it.
  \item test-output for running the compiler on \texttt{compila.cmp} as
    input
  \item of course, all code needed to run your solution 
  \item the Java-classes for the syntax-tree
  \item the build-script \texttt{build.xml} (adapted)
  \end{itemize}
\end{itemize}

Of course, the old code (for lex and yacc-based parsing) is still
needed. It's not needed that both versions of the grammar, required for
oblig 1, are still supported, one working version is enough.



\section{Purpose and goal}
\label{sec:x}

The goal of the task is to collect more practical experience implementing a
compiler, in particular, a taste of phases after parsing. It's only a
taste, as we don't have the time to get a full-scale compiler on its
feet. The language we are compiling is (as before) described in the
\emph{compila 20 language specification.} This time, also the later sections
about type checking etc, what were irrelevant for oblig 1, specify the
scope of the task as far as the language features are concerned.


Testing becomes more important than in oblig 1. It's \emph{necessary} that
a solution is equipped

\begin{quote}
  with ``automatic test-cases''
\end{quote}
That can be done (as before) via ant targets. Those tests have to be
executable on the RHEL linux pool at the university.\footnote{That should
  actually not be a big restriction, as Java (and the task) is to a big
  extent platform independent (``write once, run everywhere''
  \ldots). Nonetheless: Based on experience with the earlier years (this
  year actually no problems occured): it's advised to make this ``test''
  setup early on (not after the deadline), to design the code \emph{with
    the goal that it runs also at a different place than one's own
    platform} and to test that this goal is actually met. The reason for
  that ``testability'' requirement is that correction will again not be
  based on reading much code from my side, but in first approximation:
  running the test. In that sense, it's also not of primary importance,
  whether it's \texttt{ant} or perhaps \emph{make} or some
  script. Important is, that I can execute it by invoking a simple command.
  I don't have the time to figure out how one particular solution is
  configured, started, etc. I don't even want to look around and try
  whether I find a \texttt{main} method somewhere\ldots }


\section{Tools}
%\label{sec:tools}


The tools are basically the same as for the previous oblig, and typically
you will continue anyway with the previous set-up. 




\section{Task more specifically: Type checking and code generation}
\label{sec:task-more-spec}



The task is to extend the parser and AST generation with type checking and
code generation. The rules governing the type checking and other
restrictions are described in the language specification already (in the
later sections). The ``semantics'' is \emph{not} specified, but the
language is so simple that it should basically be clear what a compila
program is supposed to do.

The target ``platform'' is described in a separate document (which was
already made available as part of the git-repos). It's also browsable under


\begin{quote}
  \url{https://github.uio.no/msteffen/compila/tree/master/doc/bytecodeinterpreter}
\end{quote}





\subsection*{Tests}
\label{sec:tests}

The tests that need to be successfully run for oblig 2 are
\begin{enumerate}
\item testing the type checker resp. semantic analysis
\item testing the code coge generator
\end{enumerate}


The tests are located as follows relative to the
\texttt{oblig2patch}-directory

\begin{verbatim}
   ./src/tests/semanticanalysis/
\end{verbatim}


\begin{verbatim}
   ./src/tests/fullprograms/runme.cmp
\end{verbatim}


You may place them inside the \emph{you} src-directory (and add then to
\emph{your} repository).

\subsection*{Patch}
\label{sec:patch}


Obtain the patch (as zip-archive) under 

\begin{quote}
  \url{https://github.uio.no/msteffen/compila/tree/master/oblig2patch/oblig2patch.zip}
\end{quote}

or via an updated clone of the course repos. Read also the ~Readme.org~
there, there is more info about how to start. Note: when you cloned or
downloaded the repository in perhaps March, there had been already some
\emph{oblig2patch} as part of the repos. Also that should be more or less
usable, it reflects the 2020 version. Not much has changed since then
(mostly the documentation like this file handout here and the slides, the
actualy ``content'' is basically the same




%\bibliographystyle{apalike}
%{\small
% \bibliography{string,semantics,crossref}
%% \bibliography{extracted}
% }




%\section{Resources}
%\label{sec:resources}







\end{document}

Model Checking Cache Coherence Protocols for Distributed File Systems

%%% Local Variables: 
%%% mode: latex
%%% TeX-master: t
%%% End: 

%%% Local Variables: 
%%% mode: latex
%%% TeX-master: t
%%% End: 
